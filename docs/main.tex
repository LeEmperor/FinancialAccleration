

\documentclass[12pt]{article}


\usepackage{parskip}
\usepackage{setspace}

\onehalfspacing

\begin{document}


\begin{titlepage}
  \centering
  {\Huge\bfseries Dorm Room \par High-Frequency Trading \par} % Title
  \vspace{.5cm}
  {\Large University of Florida \par}
  \vspace{2cm}
  {\Large Concept Exploratory \par}
  \vspace{1.5cm}
  {\large Bo Purtell \par}
  \vspace{2cm}
  % {\large \today \par}
  {\large August 27, 2025\par}
\end{titlepage}

% --------------------------------------------------------
\section*{Executive Summary}
% 4 parts, each with unique challenges and implementation expertises

\clearpage
% --------------------------------------------------------


% --------------------------------------------------------
\section*{Motivations}
Quant jobs make alot of money. I like money. 
I also like hard things \emph{wink.}

And people pay alot of money for hard things to be done. Thus is follows that if I follow the money, I will find hard things to do.

So who pays alot? Banks pay alot.
But why would a bank want an engineer?
You'd be surprised.

Quant traders are a subset of traders that use advanced math and computing algorithms to power their decision making.
But with what tools and data do they make these trades? Their trusty computer models and algorithms.
\\ This of course opens the realm of low-latency systems in finance - most often engineered in C++.
These systems are known to be insanely fast.

But do you know what's faster than software? \\ Hardware.

So this set's the stage: Designing ultra-fast systems that deal with high levels of precision and insane throughput rates.
Seconds cost millions. 
\newline
Our Mission: 
\begin{itemize}
  \item Process data from the exchange
  \item Make decisions as fast as possible
  \item Send our decisions (buy/sell orders) back to the exchange
  \item Optional: Try not to fuck up too much: Mistakes cost billions 
\end{itemize}

Insanely difficult.
Niche of the niche.
Razorthin margins.
\\ So how do we go about learning how these systems work? 
\newline
\\ \textbf{\emph{We make one}}.

\clearpage
% --------------------------------------------------------


% --------------------------------------------------------
\section*{Problem Statement}
% how to do HFT stuff without HFT stuff

Given the dominance of special firms that make use of this framework to improve their edge through specialized hardware, optimized networks, and low-latency algorithms, we must now find a way to mimic this infastructure. 

While this setup may sound easy on paper, there is a significant barrier to entry for smaller entities including small firms, independent developers, and academic researchers like ourselves who lack expensive resources.

This presents three (3) key challenges: 
\begin{enumerate}
  \item \textbf{Technical Complexity} - High-Frequency Trading (HFT) platforms require highly specialized engineering skills across networking, FPGA design, and distrbuted systems. On their own these are already niche enough subjects, but when combined result in nothing short of a steep learning curve before even basic trading strategies can be protoyped.
  \item \textbf{Resource Inequality} - Colocated servers, FPGA accelerators/boards, and expensive networking equipment  and expensive networking equipment  and expensive networking equipment  and expensive enterprise-grade networking equipment limit participation in these markets to well-funded institutions - often excluding smaller players.
  \item \textbf{Data Scarcity} - Getting ahold of generic stock-market data is easy enough with frameworks such as PolygonAPI and YahooFinance, yet sufficiently granular tick-level data is prohibitively expensive or outright unavailable as many providers provide delayed or aggregated feeds leaving sub-millisecond strategies nearly impossible. 
\end{enumerate}

As a result, innovation is slowed; and promising ideas from individuals or startups are often left unexplored.
But do we really care? \\ No. We just want to prove we have the skills to run with the big leagues.
\\
So it's \textbf{\emph{hard}}, it's \textbf{\emph{expensive}}, and premium gasoline is \textbf{\emph{scarce}}. What to do...
\clearpage
% --------------------------------------------------------


% --------------------------------------------------------
\section*{Approach}

Well we wouldn't be here if we thought that some part of this \emph{wasn't} \textbf{un}achieveable.

\section{Breaking the Problem Down}
The first challenge is technical complexity. These systems have little margin for error, and exist as one of the few cases where that \emph{small} little change or difference makes 

\clearpage
% --------------------------------------------------------


% --------------------------------------------------------
\section*{Technical Background}

SerDes and >1 Gb Transmission


\clearpage
% --------------------------------------------------------


% --------------------------------------------------------
\section*{Applications}

\clearpage
% --------------------------------------------------------


% --------------------------------------------------------
\section*{Next Steps}

\clearpage
% --------------------------------------------------------


% --------------------------------------------------------
\section*{References}

\clearpage
% --------------------------------------------------------

\end{document}

